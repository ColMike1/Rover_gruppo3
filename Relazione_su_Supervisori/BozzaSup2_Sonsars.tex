\documentclass[12pt, a4paper]{article} 
\usepackage{graphicx}
\usepackage{float}
\usepackage{subcaption}
\usepackage{makecell}
\usepackage{enumitem} 
\usepackage{array} % Per definire meglio le colonne
\usepackage{longtable} % Per tabelle che si estendono su più pagine
\usepackage{tabularx} % Per colonne auto-adattanti
\usepackage{multirow}
\usepackage{amsmath}
\graphicspath{ {./images/} }
\setcounter{tocdepth}{4}    % Permette a \paragraph di apparire nell'indice
%\setcounter{secnumdepth}{4} % Opzionale: numera anche i paragrafi (es. 1.1.1.1)

\title{Supervisori}
\date{Febbraio 2026}




%%%%%%%%%%%%%%%%%%%%%%%%%%%%%%%%%%%%%%%%%%%%%%%%%%%%%%%%%%%%%%%%%%%%%%
\begin{document}
\maketitle
\newpage
\tableofcontents
% \listoffigures








\newpage
\section{Condizioni di funzionamento}

Il rover è dotato di due schede di controllo, denominate Board 1 e Board 2, che collaborano per garantire il funzionamento del sistema. In Figura~\ref{fig:architettura} è mostrata l'architettura generale del sistema.

\begin{figure}[H]
    \centering
    \includegraphics[width=1\textwidth]{images/funzionamento_generale/specifiche/architettura}
    \caption{Architettura generale del sistema.}
    \label{fig:architettura}
\end{figure}

\subsection{Condizioni di funzionamento nominale}
In condizioni di funzionamento nominale la Board 1 è considerata il $Master$, occupandosi dell'attuazione dei motori, inviando tramite UART l'ingresso di controllo. La Board 2 è considerata $Slave$ in quanto in condizioni nominali
non comanda i motori. Questi ultimi possono raggiungere una velocità massima di $160 RPM$.
Il riferimento di velocità lineare o di rotazione è calcolato in base ai comandi utente inviati attraverso un joistick tramite protocollo \textit{Bluethooth} ad un $ESP32$ e ricevuti dalla Board 2 tramite protocollo $I2C$.
La Board 2 quindi invia i comandi utente ricevuti alla Board 1, che si occupa di eseguirli. 
\paragraph{Rilevamento ostacoli}
I riferimenti di velocità sono influenzati da condizioni di sicurezza quali la presenza di ostacoli statici o in movimento.
Infatti, in presenza di ostacoli statici a meno di $70$ cm, il supervisore di Board 2 forza un comando di $CMD\_ESTOP$, mentre in presenza di ostacoli in movimento a distanza compresa tra $75$ e $150$ cm, il supervisore di Board 2, che si occupa 
di rilevare ostacoli, attiva delle procedure di deviazione o di aggiramento, come $CMD\_AVOID\_LEFT$ o $CMD\_GO\_LEFT$. Essendo che in condizioni nominali Board 2 non attua, questi comandi vengono inviati dalla Board 2, tramite la comunicazione UART, alla Board 1 che si occupa di eseguirli.



\subsection{Condizioni di funzionamento critico o degradato}

Il ciclo di vita delle due board prevede la gestione di stati operativi $non$ nominali:
\begin{itemize}
    \item \textbf{Modalità degradata}: il sistema mantiene la funzionalità ma con velocità ridotte del $50\%$.
    \item \textbf{Modalità critica}: il sistema si ferma, interrompendo l'esecuzione dei comandi utente per motivi di sicurezza.
\end{itemize}


\paragraph{Meccanismo di rilevazione dei guasti}
Il passaggio a queste modalità avviene monitorando la frequenza di aggiornamento dei dati provenienti da sensori e periferiche. Ogni Board interroga i propri componenti e considera la lettura "corretta" solo quando la funzione software dedicata restituisce uno stato di successo.
Ogni volta che viene convalidata una lettura, il sistema aggiorna un timestamp di riferimento, che rappresenta l'ultimo momento in cui i dati sono stati considerati affidabili.
Invece di basarsi su un singolo ritardo isolato, il sistema calcola il periodo medio di ricezione tramite una media mobile degli ultimi 10 intervalli temporali. Questo approccio evita che piccoli ritardi casuali o singole letture perse causino inutili cambi di modalità.

Ogni Board agisce quindi in base al valore medio calcolato:
\begin{itemize}
\item \textbf{Soglia di allerta}: se il periodo medio di aggiornamento supera i $40\,\mathrm{ms}$, il sistema passa in modalità degradata, segnalando un rallentamento eccessivo dei dati.
\item \textbf{Soglia massima}: se la media degli intervalli supera i $120\,\mathrm{ms}$, il guasto è considerato critico; il sistema viene bloccato per garantire la sicurezza del rover e dell'utente.
\end{itemize}

Per alcune periferiche specifiche, come gli encoder, la sola analisi temporale non è sufficiente;
in questi casi viene applicata una logica di diagnostica più avanzata per verificare la validità del segnale.


\paragraph{Encoder non funzionanti}
Il rover è progettato per continuare a funzionare anche in presenza di guasti agli encoder delle ruote.
\begin{figure}[H]
    \centering
    \includegraphics[width=1\textwidth]{images/funzionamento_generale/specifiche/Encoder_faults}
    \caption{Logica di sostituzione in presenza di encoder non funzionanti.}
    \label{fig:encoder_fault}
\end{figure}
Se uno o più encoder si rompono, il sistema stima la velocità mancante usando quella delle ruote sane. La logica di sostituzione è la seguente:
\begin{itemize}
    \item \textit{1 Fallimento}: Se si rompe l'encoder anteriore sinistro, il sistema "copia" il valore di quello posteriore sinistro.
    \item \textit{2 Fallimenti}: Se si rompe un intero lato, copia i valori dall'altro lato.
    \item \textit{3 Fallimenti}: Tutte le ruote assumono il valore dell'unica ruota superstite.
\end{itemize}
La rilevazione di un encoder non funzionante avviene attraverso una logica che monitora la potenza erogata al motore e la velocità rilevata dagli encoder. Se si alimenta un motore ma gli encoder non 
rilevano movimenti (velocità calcolata è nulla) per 20 cicli di esecuzione, si imposta la variabile $has\_no\_feedback(i)$ associata al motore di riferimento a $1$, indicando che l'encoder è considerato non funzionante
e attivando la logica di sostituzione descritta sopra. In questo caso viene segnalato un guasto di tipo $Degraded$ al supervisore di Board 1, che decide di limitare la potenza erogata ai motori. In particolare
si applica il $50\%$ di potenza massima alle ruote, in modo da ridurre il rischio di slittamento o perdita di controllo dovuto alla mancanza di feedback dagli encoder.

\paragraph{Coerenza di movimento del rover}
La Board 2 monitora uno stato di coerenza tra la rotazione stimata dalla lettura della velocità dei motori (calcolata dagli encoder) e quella misurata dall'IMU. Se ci sono incoerenze, il supervisore di Board 2 
imposta un flag $isMotionConsistent$ a 0, indicando che il movimento del rover non è affidabile.
Nel caso in cui, oltre a questa condizione di incoerenza, ci si trovi anche nella condizione di degrado di uno o più encoder e si riceva anche un segnale di incoerenza da parte dell'IMU, il supervisore di Board 1 decide di fermare il rover.

\paragraph{Board 2 non riceve}

Se la Board 2 non riceve dati dalla Board 1 per un periodo di tempo superiore a 120 ms, la Board 2 decide di inviare il comando di attuazione al rele.
Tuttavia, dato che Board 1 continua a riceve da Board 2, non cede i privilegi di attuazione, quindi Board 2 si limita ad inviare comandi di attuazione in open loop, basandosi esclusivamente sui comandi utente ricevuti 
e sui dati dei sensori, ma senza effettivamente attuare.
\begin{figure}[H]
    \centering
    \includegraphics[width=1\textwidth]{images/funzionamento_generale/specifiche/Board2_non_riceve}
    \caption{Board 2 non riceve dati dalla Board 1.}
    \label{fig:board2_non_riceve}
\end{figure}

Un aspetto da notare è che la Board 2, come descritto nel paragrafo \textit{Coerenza del movimento del rover}, continua a verificare che l'angolo di rotazione stimato dalla velocità delle ruote sia coerente con quello misurato dall'IMU.
Questa coerenza, nel caso di mancata ricezione dalla Board 1, viene meno in quanto
la stima dell'angolo di rotazione è basata sull'ultimo dato di velocità delle ruote che ovviamente cambia in quanto l'utente può ancora comandare il rover. Essendo che la rotazione del rover può essere possibile Board 2 imposterà il flag 
$isMotionConsistent$ a 0, indicando che il movimento del rover non è affidabile. 

A differenza di come descritto nel paragrafo \textit{Encoder non funzionanti}, nel caso in cui uno o più encoder si rompano, la Board 1 deciderà di attuare uno stop di emergenza, perchè avrà uno stato di degrado di uno o più encoder 
in aggiunta ad un segnale di incoerenza del movimento del rover segnalato da Board 2. 


\paragraph{Board 1 non riceve}

Se la Board 1 non riceve dati dalla Board 2 per un periodo di tempo superiore a 120 ms, la Board 1 decide di cedere i privilegi di attuazione alla Board 2, quindi aprendo il rele e lasciando che Board 2 possa attuare i comandi di
movimento. 
\begin{figure}[H]
    \centering
    \includegraphics[width=1\textwidth]{images/funzionamento_generale/specifiche/Board1_non_riceve}
    \caption{Board 1 non riceve dati dalla Board 2.}
    \label{fig:board1_non_riceve}
\end{figure}
Il controllo di Board 2 avviene in open loop, basandosi esclusivamente sui comandi utente ricevuti e sui dati dei sensori, senza feedback diretto sulla velocità effettiva del rover.
Essendo questa una modalità di funzionamento degradata, la velocità del rover viene dimezzata.






















\newpage
\section{Funzionamento generale del software di ciascuna board}
\subsection{Interazione tra Task e Meccanismo di Snapshot}
Ciascuna board gestisce sette task, ognuno dei quali attinge, secondo necessità, alle informazioni richieste dalla propria logica di controllo tramite gli snapshot, ovvero 
variabili globali condivise. Si consideri, ad esempio, la Figura~\ref{fig:funzionamento_generale} che illustra il funzionamento dei task \textit{SupervisorB1} e \textit{Transmit} che vengono eseguiti sulla Board 1. 
Al termine della propria esecuzione, ogni task può aggiornare tali snapshot, rendendo i nuovi dati disponibili agli altri task. Nell'esempio, il task \textit{SupervisorB1} 
aggiorna lo snapshot \textbf{supervisor\_snapshot}, che viene poi utilizzato dal task \textit{Transmit} per l'invio dei dati verso la Board2. Allo stesso modo, la Board 2 tiene legge
i scrive i propri snapshot, come illustrato nella Figura~\ref{fig:funzionamento_generale_b2}.

Ogni snapshot integra, oltre ai dati specifici, due variabili di monitoraggio temporale:

\begin{itemize} 
    \item \textit{data\_last\_valid\_ms}: indica l'istante temporale dell'ultimo aggiornamento dei dati ritenuto valido (ad esempio, l'ultima lettura coerente ricevuta dagli encoder). 
    \item \textit{task\_last\_run\_ms}: riporta l'ultimo istante in cui il task è stato effettivamente eseguito. 
\end{itemize}


\begin{figure}[H]
    \centering
    \includegraphics[width=1\textwidth]{images/funzionamento_generale/funzGener_B1}
    \caption{Funzionamento generale del software di Board 1.}
    \label{fig:funzionamento_generale}
\end{figure}

\begin{figure}[H]
    \centering
    \includegraphics[width=1\textwidth]{images/funzionamento_generale/funzGener_B2}
    \caption{Funzionamento generale del software di Board 2.}
    \label{fig:funzionamento_generale_b2}
\end{figure}




\subsection{Funzionamento task di Board 2}
\subsubsection{Task: acquisizione distanza dagli ostacoli}
La Board 2 è equipaggiata con tre sensori ad ultrasuoni \textbf{HC-SR04} per il rilevamento di ostacoli. 
Questi sono disposti a 45° l'uno dall'altro, come mostrato in Figura~\ref{fig:sensori}.


\begin{figure}[ht]
    \centering
    \includegraphics[width=0.6\textwidth]{images/funzionamento_generale/Sonars/disposizioneSonar} 
    \caption{Disposizione dei sensori sulla Board 2.}
    \label{fig:sensori}
\end{figure}

Ogni sensore emette onde sonore ad alta frequenza e produce segnali di tipo onda quadra la cui durata è proporzionale all'ostacolo rilevato. 
\begin{figure}[H]
    \centering
    \includegraphics[width=0.7\textwidth]{images/funzionamento_generale/Sonars/segnale_sonar_3m}
    \caption{Segnale generato dal sensore HC-SR04 in presenza di un ostacolo a 3 metri di distanza.}
    \label{fig:segnale_sonar}
\end{figure}
La board2, rilevando i fronti di salita e discesa, può misurare l'intervallo tra i due fronti e utilizzare questa informazione per calcolare
la distanza dall'ostacolo e prendere decisioni appropriate per evitare collisioni.

\paragraph{Utilizzo DMA per la lettura dei segnali}
Per ottimizzare la lettura dei segnali dai sensori ad ultrasuoni, la Board 2 utilizza il Direct Memory Access (DMA).
Il DMA consente di trasferire i dati direttamente tra la periferica (i sensori ad ultrasuoni) e la memoria, senza l'intervento della CPU.
Il timer utilizzato è il $Timer 1$, con i canali $1$, $2$ e $3$ configurati in modalità \textit{input capture} per catturare i fronti di salita e discesa generati dai tre sensori.
\begin{figure}[H]
    \centering
    \includegraphics[width=0.7\textwidth]{images/funzionamento_generale/Sonars/configurazione_timIC_sonar}
    \caption{Configurazione del DMA per la lettura dei segnali dai sensori.}
    \label{fig:dma_sonar}
\end{figure}

Ogni canale del DMA è configurato in modalità interrupt, permettendo, alla fine della rilevazione dei due fronti (salita e discesa), di eseguire una $Callback$ che imposta dei flag a 1. 
Questo flag indica che i fronti sono stati rilevati e che la distanza dall'ostacolo può essere calcolata. In totale vengono eseguite solo $3$ callback, attivate solo quando uno specifico 
canale DMA ha terminato la lettura di entrambi i fronti. La Figura~\ref{fig:callback_sonar} mostra un esempio di callback eseguita al termine della rilevazione dei fronti.
\begin{figure}[H]
    \centering
    \includegraphics[width=0.7\textwidth]{images/funzionamento_generale/Sonars/callback_sonar}
    \caption{Callback eseguita al termine della rilevazione dei fronti.}
    \label{fig:callback_sonar}
\end{figure}

Alla fine della lettura, il task aggiorna lo snapshot \textbf{SonarsSnapshot} con le distanze rilevate dai tre sensori, come mostrato in Figura~\ref{fig:sonars_snapshot}.
\begin{figure}[H]
    \centering
    \includegraphics[width=0.7\textwidth]{images/funzionamento_generale/Sonars/sonars_snapshot}
    \caption{Struttura dati dello snapshot SonarsSnapshot.}
    \label{fig:sonars_snapshot}
\end{figure}





\subsubsection{Task: lettura comandi utente}
La Board 2 riceve i comandi utente provenienti da un joistick che comunica con un'ESP32 tramite Bluetooth. La Board 2, attraverso un task dedicato, riceve i comandi provenienti dalla $ESP32$
con il protocollo \textbf{I2C}e li integra nello snapshot \textbf{BleControllerSnapshot}. In Figura~\ref{task_step_BLE} è mostrata lo step del task dedicato alla ricezione dei comandi utente, 
in cui viene eseguita la lettura dei dati provenienti dalla $ESP32$ e l'aggiornamento dello snapshot \textbf{BleControllerSnapshot}. 

\begin{figure}[H]
    \centering
    \includegraphics[width=0.8\textwidth]{images/funzionamento_generale/user_command/task_step_BLE}
    \caption{Step del task dedicato alla ricezione dei comandi utente.}
    \label{task_step_BLE}
\end{figure}

La funzione $BleController\_I2C\_ReadFrame$ si occupa di leggere i dati provenienti dalla $ESP32$ e di restituirli in una struttura dati, che viene poi utilizzata per aggiornare lo snapshot \textbf{BleControllerSnapshot}.

In Figura~\ref{fig:ble_snapshot} è mostrata la struttura dati dello snapshot \textbf{BleControllerSnapshot},
in cui sono presenti i comandi utente normalizzati ($x\_norm$ e $y\_norm$).

\begin{figure}[H]
    \centering
    \includegraphics[width=0.7\textwidth]{images/funzionamento_generale/user_command/BLE_snapshot}
    \caption{Struttura dati dello snapshot BleControllerSnapshot.}
    \label{fig:ble_snapshot}
\end{figure}

\subsubsection{Task: lettura giroscopio}
La Board 2 è equipaggiata con un sensore IMU \textbf{MPU-6050}, che integra un accelerometro e un giroscopio a 3 assi. Poiché il sensore non fornisce direttamente l'orientamento assoluto, l'angolo di yaw
(la rotazione del rover attorno al suo asse verticale) viene calcolato nel firmware integrando nel tempo i dati della velocità angolare provenienti dal giroscopio.
All'accensione, il sistema esegue una calibrazione per impostare lo zero relativo alla direzione di avvio. La comunicazione con l'IMU avviene tramite protocollo I2C, e i dati elaborati aggiornano
ciclicamente lo snapshot IMUSnapshot (Figura~\ref{imu_snapshot}).

\begin{figure}[H]
    \centering
    \includegraphics[width=0.7\textwidth]{images/funzionamento_generale/IMU/IMU_snapshot}
    \caption{Struttura dati dello snapshot IMUSnapshot.}
    \label{imu_snapshot}
\end{figure}

\subsubsection{Task: supervisore Board 2}
Il cuore logico della Board 2 è rappresentato dal task del supervisore, un processo che implementa una macchina a stati complessa generata tramite Simulink Stateflow. 
Il task opera come un arbitro di sicurezza tra i comandi utente e l'ambiente circostante, garantendo che il rover si muova nella direzione voluta dall'utente a meno di 
possibili ostacoli statici o in movimento.

In Figura~\ref{fig:supervisorB2_task} è mostrato lo step del task del supervisore, in cui vengono evidenziati gli input e gli output.


\begin{figure}[H]
    \centering
    \includegraphics[width=0.8\textwidth]{images/funzionamento_generale/TaskSupervisoreB2/task_step_supervisorB2}
    \caption{Step del task del supervisore della Board 2.}
    \label{fig:supervisorB2_task}  
\end{figure}

Ad ogni ciclo, il task acquisisce i dati dai sensori locali (\textbf{Sonar}, \textbf{IMU}, \textbf{Comandi utente}) e le informazioni provenienti dalla Board 1 tramite lo snapshot \textbf{RxSnapshot}.
Il sistema valuta l'integrità dei dati e la persistenza della comunicazione con la Board 1. In base alla latenza dei messaggi e alla coerenza dei sensori, 
il supervisore può far transitare il rover in stati di operatività $Degraded$ o $Critical$. Utilizzando i dati dei tre sonar frontali, il supervisore implementa algoritmi di protezione, infatti,
se viene rilevato un ostacolo a distanza critica ($<70$ cm), il sistema forza un comando di $CMD\_ESTOP$.
In caso di ostacoli in movimento (range $75-150$ cm), il supervisore attiva delle procedure di deviazione ($CMD\_GO\_LEFT$, $CMD\_GO\_RIGHT$) o di aggiramento ($CMD\_AVOID$), calcolando i nuovi target 
di angolo yaw necessari per direzionare il rover verso l'assenza di ostacoli.

Nel paragrafo "Coerenza di movimento del rover" si è parlato della capacità della Board 2 di monitorare la coerenza del movimento del rover. In effetti, il supervisore esegue un test di coerenza
tra la rotazione stimata dalla lettura della velocità dei motori e quella misurata dall'IMU. Questa logica permette di rilevare 
anomalie meccaniche (come lo slittamento di una ruota o il bloccaggio di un motore) o anomalie dei sensori IMU ed encoder. Se ci sono incoerenze, il supervisore di Board 2 imposta un flag $isMotionConsistent$ a 0.

Infine, in condizioni nominali, la Board 2 non invia comandi di attuazione ai motori, lasciando che sia la Board 1 ad attuare. Tuttavia, il supervisore consente alla Board 2 di attuare se si verifica una di queste 
condizioni:
\begin{itemize}
    \item Viene rilevato un timeout critico nella ricezione dei dati dalla Board 1 ($>120$ ms).
    \item Il supervisore della Board 1 funziona in maniera discontinua.
    \item Board 1 invia a Board 2 l'autorizzazione ad attuare.
\end{itemize}

E' da notare che l'attuazione passa attraverso il rele comandato da Board 1, quindi dovrà essere sempre Board 1 ad aprirlo per consentire al segnale UART di Board 2 di arrivare ai driver dei motori.

In Figura~\ref{fig:b2_actuation} è mostrata la $actuation\_step$ di Board 2 in funzione della variabile $autorized\_to\_send\_command$ in uscita dal supervisore.

\begin{figure}[H]
    \centering
    \includegraphics[width=0.8\textwidth]{images/funzionamento_generale/TaskSupervisoreB2/B2Actuation}
    \caption{Step di attuazione di Board 2 in funzione della variabile autorized\_to\_send\_command.}
    \label{fig:b2_actuation}
\end{figure}

Data l'assenza degli encoder, l'attuazione di Board 2 dovrà essere in open loop, basandosi esclusivamente sui comandi utente ricevuti e sui dati dei sensori, senza feedback diretto sulla velocità effettiva del rover.
La funzione di attuazione trasforma i riferimenti di velocità del supervisore in segnali di potenza per i motori. Il sistema opera in anello aperto (open-loop), convertendo i giri al minuto (RPM) desiderati in tensione elettrica
Infine, i comandi vengono inviati al driver di potenza Sabertooth tramite una scalatura lineare del voltaggio in uscita, con un range di $[-12V; +12V]$ corrispondente a $[-MAX\_RPM; +MAX\_RPM]$.

Lo snapshot del supervisore di Board 2 è mostrato in Figura~\ref{fig:b2_snapshot}:

\begin{figure}[H]
    \centering
    \includegraphics[width=0.8\textwidth]{images/funzionamento_generale/TaskSupervisoreB2/snapshot_supervisoreB2}
    \caption{Snapshot del supervisore di Board 2}
    \label{fig:b2_snapshot}
\end{figure}




\subsubsection{Task: log dei dati per il debug}
La Board 2 integra un task dedicato alla stampa dei dati per scopi di debug. Questo task, eseguito a cadenza regolare,
acquisisce gli snapshot:
\begin{itemize}
    \item \textbf{RxSnapshot}
    \item \textbf{BleControllerSnapshot}
    \item \textbf{SonarsSnapshot}
    \item \textbf{IMUSnapshot}
\end{itemize}
e stampa i dati più rilevanti su console. Questa funzionalità è fondamentale per monitorare lo stato del sistema durante le fasi di sviluppo e test, permettendo di identificare rapidamente eventuali anomalie o comportamenti imprevisti.
Lo step del task è mostrato in Figura~\ref{fig:log_task}, in cui vengono evidenziati i dati acquisiti e stampati su console.

\begin{figure}[H]
    \centering
    \includegraphics[width=0.8\textwidth]{images/funzionamento_generale/LogBoard2/LogTask}
    \caption{Step del task dedicato alla stampa dei dati per scopi di debug.}
    \label{fig:log_task}
\end{figure}


%%%%%%%%%%%%%%%%%%%%%%%%%%%%%%%%%
%%%%%%%%%%%%%%%%%%%%%%%%%%%%%%%%%
%%%%%%%%%%%%%%%%%%%%%%%%%%%%%%%%%
%%%%%%%%%%%%%%%%%%%%%%%%%%%%%%%%%
%%%%%%%%%%%%%%%%%%%%%%%%%%%%%%%%%
%%%%%%%%%%%%%%%%%%%%%%%%%%%%%%%%%
%%%%%%%%%%%%%%%%%%%%%%%%%%%%%%%%%
%%%%%%%%%%%%%%%%%%%%%%%%%%%%%%%%%
%%%%%%%%%%%%%%%%%%%%%%%%%%%%%%%%%
%%%%%%%%%%%%%%%%%%%%%%%%%%%%%%%%%
%%%%%%%%%%%%%%%%%%%%%%%%%%%%%%%%%
%%%%%%%%%%%%%%%%%%%%%%%%%%%%%%%%%

\subsection{Funzionamento task di Board 1}
\subsubsection{Task: lettura batteria e temperatura}
La Board 1 è equipaggiata con sensori per il monitoraggio della temperatura e della batteria. Un task dedicato esegue ciclicamente la lettura di questi sensori, aggiornando lo snapshot \textbf{BoardHealthSnapshot} con i valori rilevati.

In Figura~\ref{fig:health_task} è mostrato lo step del task dedicato alla lettura della batteria e temperatura, in cui vengono evidenziati i dati acquisiti e aggiornati nello snapshot \textbf{BoardHealthSnapshot}.

\begin{figure}[H]
    \centering
    \includegraphics[width=0.8\textwidth]{images/funzionamento_generale/BatTemp/BoardHelathTask}
    \caption{Step del task dedicato alla lettura della batteria e temperatura.}
    \label{fig:health_task}
\end{figure}

Per garantire l'affidabilità a lungo termine, il sistema integra un modulo di diagnostica basato su convertitori Analogico-Digitali (ADC) che monitorano costantemente i parametri vitali della scheda. Questo modulo si occupa di due aspetti critici:

\begin{itemize}
\item \textbf{Gestione della Batteria:} Il sistema legge la tensione della batteria attraverso un partitore resistivo. Poiché la scarica di una batteria non è lineare, il software utilizza una \textit{Look-Up Table} (LUT) e un'interpolazione
                                        lineare per convertire i Volt in una percentuale di carica residua (0-100\%). Ciò fornisce all'utente un'indicazione realistica dell'autonomia residua.
\item \textbf{Monitoraggio Termico:}    Utilizzando il sensore di temperatura interno al microcontrollore STM32G4, il sistema calcola la temperatura operativa della logica di controllo. Il calcolo sfrutta i dati di calibrazione salvati in fabbrica
                                        dal produttore per garantire la massima precisione.
\end{itemize}

Per evitare letture errate dovute a disturbi elettrici o picchi temporanei, i dati non vengono usati così come sono, ma vengono processati attraverso una media mobile. 
Questo filtro "ammorbidisce" le letture su una finestra di 10 campioni, garantendo che i valori visualizzati siano stabili e puliti. 
Infine, il modulo verifica che i valori rientrino in range di sicurezza (es. 7V - 15V per la batteria), segnalando errori di sistema qualora i parametri diventino critici

\subsubsection{Task: supervisore Board 1}
Il supervisore della Board 1 è implementato tramite una macchina a stati sviluppata in Simulink Stateflow. Questo task ha il compito di supervisionare l'integrità dell'hardware e di assegnare i riferimenti di velocità lineare e di rotazione.

In Figura~\ref{fig:supervisorB1_task} è illustrato lo step del task del supervisore della Board 1, con il dettaglio delle interfacce di input e output.

\begin{figure}[H]
\centering
\includegraphics[width=0.8\textwidth]{images/funzionamento_generale/TaskSupervisoreB1/task_step_supervisorB1}
\caption{Step del task del supervisore della Board 1.}
\label{fig:supervisorB1_task}

\end{figure}

Il task elabora in tempo reale i dati provenienti dalla \textbf{Board Health} (stato di batteria e temperatura), i feedback degli \textbf{Encoder} motori e i pacchetti dati ricevuti dalla Board 2.


La logica di controllo è strutturata su tre pilastri fondamentali:
\begin{itemize}
\item \textbf{Diagnostica e Fault Masking:} il sistema valuta l'integrità dei dati e la persistenza della comunicazione con la Board 2. In base alla latenza dei messaggi e alla coerenza dei sensori, può
                                            far transitare il rover in stati di operatività $Degraded$ o $Critical$. Questi stati permettono al supervisore di inibire manore o di limitare la potenza alle ruote
                                            in caso di anomalie, garantendo maggiore protezione. 

\item \textbf{Gestione delle Manovre Complesse:} A differenza della Board 2, focalizzata sull'evitamento ostacoli, la Board 1 gestisce comandi di alto livello come la \textit{Rotazione a 180 gradi} assistita e la procedura di \textit{Emergency Stop}, 
                                            monitorando l'arresto effettivo delle ruote tramite i sensori di velocità.
\item \textbf{Arbitraggio dell'Attuazione:} Il supervisore valuta costantemente l'affidabilità della comunicazione inter-board. Qualora rilevi un'instabilità nella comunicazione con l'altra scheda, è in grado di cedere i privilegi di attuazione alla Board 2,
                                            realizzando una strategia di ridonanza funzionale che aumenta la resilienza del rover. I motivi per cui la Board 1 può decidere di cedere il controllo alla Board 2 quando la comunicazione con quest'ultima si interrompe o si degrada
                                            derivano dal fatto che Board 2 ha a disposizione i sonar e sopratutto il comando utente, fondamentali per decider dove andare e se deviare il percorso o addirittura fermarsi.
\end{itemize}

Lo snapshot del supervisore di Board 1 è mostrato in Figura~\ref{fig:b1_snapshot}.

\begin{figure}[H]
    
    \centering
    \includegraphics[width=0.8\textwidth]{images/funzionamento_generale/TaskSupervisoreB1/snapshot_supervisoreB1}
    \caption{Step del task del supervisore della Board 1.}
    \label{fig:b1_snapshot}

\end{figure}

\subsubsection{Task: controllo motori}
Il task di controllo presente nella Board 1 è responsabile dell'attuazione dei comandi di movimento del rover, basandosi sui riferimenti di velocità lineare e angolare forniti dal supervisore.
Il task quindi:
\begin{itemize}
    \item legge i riferimenti di velocità lineare ($v\_ref$) e angolare ($omega\_ref$) calcolati dal supervisore.
    \item legge le velocità delle ruote attraverso gli encode.
    \item applica un controllo PID per regolare la potenza erogata ai motori, al fine di raggiungere i riferimenti di velocità desiderati.
\end{itemize}
Alla lettura degli encoder per stimare la velocità delle ruote è accompagnata una logica che permette il rilevamento di un'anomalia derivante dal fatto che si sta alimentando il motore ma 
gli encoder non rilvano movimenti, quindi la velocità calcolata è nulla. In particolare, se si rileva quest'anomalia per $20$ cicli di esecuzione, si imposta la variabile $has\_no\_feedback(i)$ 
associata al motore di riferimento a $1$.

Se uno o più encoder si rompono, il rover non deve smettere di funzionare, ma deve "stimare" la velocità mancante usando quella delle ruote sane.
Il codice applica una sostituzione simmetrica:

\begin{itemize}
    \item \textit{1 Fallimento}:Se si rompe l'encoder anteriore sinistro, il sistema "copia" il valore di quello posteriore sinistro.
    \item \textit{2 Fallimenti}: Se si rompe un intero lato, copia i valori dall'altro lato.
    \item \textit{3 Fallimenti}: Tutte le ruote assumono il valore dell'unica ruota superstite.
\end{itemize}


\subsubsection{Task: log dei dati per il debug}
La Board 1 integra un task dedicato alla stampa dei dati per scopi di debug. Questo task, eseguito a cadenza regolare,
acquisisce gli snapshot:
\begin{itemize}
    \item \textbf{RxSnapshot}
    \item \textbf{EncoderSnapshot}
    \item \textbf{BoardHealthSnapshot}
\end{itemize}
e stampa i dati più rilevanti su console. Questa funzionalità è fondamentale per monitorare lo stato del sistema durante le fasi di sviluppo e test, permettendo di identificare rapidamente eventuali anomalie o comportamenti imprevisti.
Lo step del task è mostrato in Figura~\ref{fig:log_task}, in cui vengono evidenziati i dati acquisiti e stampati su console.

\begin{figure}[H]
    \centering
    \includegraphics[width=0.8\textwidth]{images/funzionamento_generale/LogBoard2/LogTask}
    \caption{Step del task dedicato alla stampa dei dati per scopi di debug.}
    \label{fig:log_task}
\end{figure}


\subsubsection{Task: led}
Per fornire un feedback immediato sullo stato operativo del sistema senza l'ausilio di terminali esterni, è stato implementato un task dedicato alla gestione della diagnostica visiva tramite LED. Il modulo \texttt{led\_task.c}
monitora le maschere di errore e le azioni correnti decise dai supervisori di entrambe le board, traducendole nel seguente protocollo di segnalazione:

\begin{table}[H]
\centering
\begin{tabular}{|l|l|l|}
\hline
\textbf{Colore LED} & \textbf{Comportamento} & \textbf{Significato Operativo} \\ \hline
Rosso & Fisso & Guasto Critico (Sistema bloccato) \\ \hline
Rosso & Lampeggiante & Guasto Degraded (Prestazioni ridotte) \\ \hline
Giallo (L/R) & Fisso & Manovra di schivata o sterzata in corso \\ \hline
Blu & Fisso & Emergency Stop attivo (Frenata di emergenza) \\ \hline
\end{tabular}
\caption{Protocollo di segnalazione visiva del rover.}
\label{tab:led_protocol}
\end{table}

Questa implementazione risulta fondamentale durante i test sul campo, poiché permette di distinguere istantaneamente tra un normale comportamento di evitamento ostacoli (LED Giallo) e un arresto dovuto a un'anomalia hardware o software (LED Rosso).


\subsection{Task di trasmissione e ricezione tra le due Board}
Le due board comunicano tra loro tramite un collegamento seriale, con presenza di task di trasmissione e ricezione su entrambe le board. Il task di trasmissione su entrambe le board ha periodo di 20 ms,
mentre il task di ricezione è di tipo event-driven, attivato dall'arrivo di nuovi dati sulla seriale. In Figura~\ref{fig:comunicazione_board21} è mostrata l'interazione tra i task di trasmissione della 
Board 2 e ricezione della Board 1, evidenziando quali sono i dati che vengono trasmessi e ricevuti tra le due board. In particolare vengono inviati gli snapshot \textbf{SupervisorSnapshot}, 
\textbf{BleControllerSnapshot} e \textbf{IMUSnapshot}. Il contenuto di questi snapshot è descritto in dettaglio nelle sezioni dedicate ai task di Board 2.

\begin{figure}[H]
    \centering
    \includegraphics[width=1\textwidth]{images/funzionamento_generale/comm_b2b1}
    \caption{Interazione tra i task di trasmissione della Board 2 e ricezione della Board 1.}
    \label{fig:comunicazione_board21}
\end{figure}

 Allo stesso modo, nella Figura~\ref{fig:comunicazione_board12} è mostrata l'interazione tra i task di trasmissione della Board 1 e ricezione della Board 2. In particolare, la Board 1 invia alla Board 2 lo
 snapshot \textbf{EncoderSnapshot} e \textbf{SupervisorSnapshot}. Il contenuto di questi snapshot è descritto in dettaglio nelle sezioni dedicate ai task di Board 1.

\begin{figure}[H]
    \centering
    \includegraphics[width=1\textwidth]{images/funzionamento_generale/comm_b1b2}
    \caption{Interazione tra i task di trasmissione della Board 1 e ricezione della Board 2.}
    \label{fig:comunicazione_board12}
\end{figure}
















































 




%%%%%%%%%%%%%%%%%%%%%%%%%%%%%%%%%%%%%%%%%%%%%%%%%%%%%%%%%%%%%%%%%%%%%%%
\newpage
\section{Supervisore Board 1}
\subsection{Panoramica generale}
Il supervisore della Board 1 è implementato come un modulo Simulink denominato \textbf{SupervisorB1}, il cui schema a blocchi è illustrato in Figura~\ref{fig:supervisorB1_block}.
\begin{figure}[H]
    \centering
    \includegraphics[width=0.8\textwidth]{images/SupervisoreB1/SupervisoreB1_simulink}
    \caption{Schema a blocchi del modulo SupervisorB1.}
    \label{fig:supervisorB1_block} 
\end{figure}
Il suo compito è quello di decidere il riferimento di velocita (\textit{v\_ref}) e di direzione (\textit{omega\_ref}) del rover, in funzione dell'elaborazione dei dati di input.
In particolare, esso è composto da quattro parti principali:
\begin{itemize}
    \item \textbf{Rilevazione Faults}: si occupa di rilevare anomalie nella ricezione dei dati da Board2, dagli encoder delle ruote, dai sensori di temperatura e batteria.
    \item \textbf{Aggregazione Fualt}: aggrega le anomalie rilevate nella parte di $rilevazione faults$ e le codifica in due maschere di errore (critica e degradata).
    \item \textbf{Decidere di far comandare la Board2}: decide se autorizzare o meno la Board2 a muovere il rover in base alle condizioni di fault rilevate.
    \item \textbf{Calcolo Riferimenti}: se Board1 attua, calcola i riferimenti di velocità lineare e angolare del rover in base ai comandi ricevuti dalla Board2 e alle condizioni di fault rilevate.
\end{itemize}
Nel seguito verranno descritti i segnali di input e output del supervisore, successivamente verranno descritte le quattro parti principali del supervisore descritte sopra.

%%%%%%%%%%%%%%%%%%%%%%%%%%%%%%%%%%%%%%%%%%%%%%%%%%%%%%%%%%%%%%%%%%%%%%%

\subsection{Input}
I segnali in input che riceve sono:


\begin{itemize}  
    \item \textbf{Board2\_Data}: è l'ultimo snapshot di ricezione ($RxSnapshot$) aggiornato dal task di ricezione di Board1. I segnali utilizzati dal supervisore di Board 1 sono:
    \begin{itemize}
        \item \textit{command}: rappresenta il comando in uscita dal supervisore della Board2, che può assumere i seguenti valori:
            \begin{figure}[H] 
                \centering
                \includegraphics[width=0.3\textwidth]{images/SupervisoreB1/CMD\_BOARD2}
                \caption{Comandi in uscita dal supervisore della Board2.}
                \label{fig:comandi_b2}
            \end{figure}
        \item \textit{x\_norm \emph{\&} y\_norm}: rappresenta il comando utente proveniente dal joistick. Così se il comando $command$ proveniente da Board 2 è $CMD\_NORMAL$ e la il supervisore di Board 1 non 
                                                  non rileva anomalie, i riferimenti di velocità lineare e angolare del rover vengono rispettivamente utilizzando 
                                                  $x\_norm$ e $y\_norm$ ($ speed\_ref = y\_norm \cdot MAX\_SPEED$ e $steering\_cmd = x\_norm \cdot MAX\_TURN$).
        \item \textit{btn1 e btn2}: segnalano la pressione di due pulsanti presenti sul joystick, che possono essere utilizzati per attivare la rotazione a 180 gradi.
        \item \textit{yaw}: rappresenta l'angolo di orientamento del rover.
        \item \textit{isMotionConsistent}: Verifica la coerenza tra la rotazione misurata dalla IMU e quella stimata dai motori, per rilevare guasti meccanici o slittamenti .
        \item \textit{critical\_mask \& degraded\_mask}: sono due mashere di errore a 8 bit calcolate da Board 2, in cui ogni bit indica la presenza 
                    di un'anomalia \textit{critica} (da cui critical\_mask) o \textit{degradata} (da cui degraded\_mask) specifica.
        \item \textit{data\_last\_valid\_ms}: rappresenta il tempo in millisecondi dell'ultimo dato valido ricevuto da Board 2, utilizzato per rilevare eventuali timeout nella ricezione.
    \end{itemize}
    \item \textbf{Board\_Health}: è l'ultimo snapshot di salute del sistema ($BoardHealthSnapshot$), in cui sono presenti i valori di temperatura e batteria. La struttura dati è la seguente:
        \begin{figure}[H]
            \centering
            \includegraphics[width=0.7\textwidth]{images/SupervisoreB1/BoardHealthSnapshot}
            \caption{Struttura dati Board\_Health.}
            \label{fig:board_health_struct}
        \end{figure}
    \item \textbf{Encoder}: è l'ultimo snapshot delle velocità delle ruote ($wheel\_speed\_rpm$) e di eventuali anomalie riscontrate in uno dei motori ($has\_no\_feedback$). La struttura dati è la seguente:
        \begin{figure}[H]
            \centering
            \includegraphics[width=0.7\textwidth]{images/SupervisoreB1/Encoder\_snapshot}
            \caption{Struttura dati Encoder.}
            \label{fig:encoder_struct}
        \end{figure}
    \item \textbf{now\_ms}: Rappresenta il tempo corrente in millisecondi.
    \item \textbf{last\_valid\_b2\_ms}: Rappresenta il tempo in millisecondi dell'ultimo volta che si è rilevato che il supervisore di Board 2 sia funzionante.
\end{itemize}

%%%%%%%%%%%%%%%%%%%%%%%%%%%%%%%%%%%%%%%%%%%%%%%%%%%%%%%%%%%%%%%%%%%%%%%

\subsection{Output}
I segnali in output che fornisce sono:
\begin{itemize}
    \item \textbf{v\_ref}: rappresenta il riferimento di velocità lineare del rover, in m/s.
    \item \textbf{omega\_ref}: rappresenta il riferimento di velocità angolare del rover, in rad/s.
    
    \item \textbf{critical\_mask \& degraded\_mask}: sono due mashere di errore a 8 bit, in cui ogni bit indica la presenza 
    di un'anomalia \textit{critica} (da cui critical\_mask) o \textit{degradata} (da cui degraded\_mask) specifica, come descritto 
    nella Tabella~\ref{tab:fault_mapping_B1}. Nella sezione di $Rilevazione Faults$ viene descritto nello specifico cosa indicano i singoli bit di queste maschere di errore.
        \begin{table}[H]
            \centering
            \begin{tabular}{lll}
            \hline
            \textbf{Bit} & \textbf{ID Segnale} & \textbf{Descrizione dell'Anomalia} \\ \hline
            0 & TEMP\_CRI/DEG & Temperatura fuori range o Timeout dall'ultima lettura corretta\\
            1 & BATT\_CRI/DEG & Tensione batteria fuori range o Timeout dall'ultima lettura corretta\\
            2 & RX\_CRI/DEG & Comunicazione con Board 2 lenta o assente \\
            3-6 & WHEEL\_CRI/DEG & Encoder non funzionanti o incoerenza totale sul movimento\\
            7 & SUP\_CRI/DEG  & Supervisore B2 lento o assente \\ \hline
            \end{tabular}
            \caption{Mappatura della maschera di errore a 8 bit}
            \label{tab:fault_mapping_B1}
        \end{table}
    \item \textbf{give\_b2\_actuation}: rappresenta un segnale booleano che indica se la Board2 deve essere autorizzata a muovere il rover.
    \item \textbf{actuate\_emergency\_stop}: rappresenta un segnale booleano che indica se il rover deve essere fermato immediatamente per evitare collisioni.
    \item \textbf{current\_action}: rappresenta un segnale che indica l'azione attualmente intrapresa dal supervisore.
        

\end{itemize}

%%%%%%%%%%%%%%%%%%%%%%%%%%%%%%%%%%%%%%%%%%%%%%%%%%%%%%%%%%%%%%%%%%%%%%%%


\subsection{Rilevazione Faults}
\begin{figure}[H]
    \centering
    \includegraphics[width=0.8\textwidth]{images/SupervisoreB1/Rilevamento_Faults}
    \caption{Chart di gestione dei faults.}
    \label{fig:gestione_faults}
\end{figure}

L'output di questo sottosistema è rappresentato da 5 segnali, ognuno dei quali indica la presenza o meno di un'anomalia \textit{critica} o
\textit{degradata} specifica proveniente dalla lettura dei segnali degli encoder, batteria, temperatura e comunicazione con la Board2. Di seguito si
 descrivono i valori che questi segnali possono assumere e le condizioni che portano a tali valori.


\begin{longtable}{ll p{5cm}} 
    \caption{Stati degradato e critico per Board 1} \label{tab:fault_table} \\
    \hline
    \textbf{OUTPUT} & \textbf{VALORE} & \textbf{SIGNIFICATO} \\
    \hline
    \endfirsthead % Cosa appare nella prima pagina

    \hline
    \textbf{OUTPUT} & \textbf{VALORE} & \textbf{SIGNIFICATO} \\
    \hline
    \endhead % Cosa appare in cima alle pagine successive (opzionale)

    
    \endfoot % Cosa appare in fondo a ogni pagina (opzionale)

    % --- Inizio dati tabella ---
    \textbf{temp\_status}       & TEMP\_HEALTH\_DEGRADED    & Quando la temperatura è nel range \newline $ ]-15; -5] \cup [55; +60[ $ \\
                                & TEMP\_HEALTH\_CRITICAL    & \begin{itemize}[nosep, leftmargin=*, after=\vspace{-\baselineskip}, before=\vspace{-0.1\baselineskip}]
                                                                \item Quando la temperatura è per almeno $4$s nel range $ ]-\infty; -15] \cup [60; +\infty[ $
                                                                \item Nel caso in cui l'intervallo di tempo dall'ultimo aggiornamento della temperatura è di $0.5$s, si ipotizza che la temperatura aumenti di
                                                                $1\,^\circ\mathrm{C}/\mathrm{s}$. Se questa sale raggiungendo un valore superiore a $65\,^\circ\mathrm{C}/\mathrm{s}$ si ha questo valore.
                                                                \end{itemize} \\\\\\
                                & TEMP\_HEALTH\_OK          & Quando la temperatura è nel range $ ]-5; 55[ $ \\ \hline 




    \textbf{batt\_status}       & BATT\_HEALTH\_DEGRADED    & Quando la percentuale di batteria è minore del $23\%$ \\\\
                                & BATT\_HEALTH\_CRITICAL    & \begin{itemize}[nosep, leftmargin=*, after=\vspace{-\baselineskip}, before=\vspace{-0.1\baselineskip}]
                                                                \item Quando la temperatura è per almeno $5\,\mathrm{s}$ nel range $ [0\%; 15\%] $
                                                                \item Nel caso in cui l'intervallo di tempo dall'ultimo aggiornamento della percentuale batteria è di $0.5\,\mathrm{s}$,
                                                                        si ipotizza che la percentuale diminuisca di $0.42\%/\mathrm{s}$. Se questa diminuisce raggiungendo un valore minore a $15\%$ si ha questo valore.
                                                                        \end{itemize} \\\\
                                & BATT\_HEALTH\_OK          & Quando la percentuale è $> 25\%$ \\ \hline
    



    \textbf{wheel\_status(i)}   & WHEEL\_DEGRADED\_ENCODER  & Viene attivato dalla funzione $has\_no\_feedback(i)$ quando si rileva un'assenza di impulsi dall'encoder nonostante
                                                              il motore sia alimentato. Questa condizione indica un guasto al sensore o un blocco 
                                                              meccanico parziale. Questa condizione permette l'attivazione di una logica di recupero (\textit{fallback}) che ricostruisce
                                                              la velocità della ruota guasta basandosi sulla lettura dei sensori integri degli altri motori.\\\\
                                & WHEEL\_CRITICAL\_MOTOR    & Questo stato viene attivato quando alla perdita del feedback dell'encoder si somma un'incoerenza tra il dato del
                                                              sensore IMU e la rotazione stimata tramite l'odometria delle ruote.
                                                              Tale discrepanza, rilevata dal controllo globale $isMotionConsistent$, indica che il sistema di fallback non è più in
                                                              grado di garantire una stima affidabile del moto. \\\\
                                & WHEEL\_OK                 & Rappresenta la condizione di pieno funzionamento del sottosistema. In questo stato, il sistema opera in assenza di 
                                                              anomalie sia a livello locale, dove la velocità calcolata dagli encoder risulta coerente con il comando impartito 
                                                              (verificato tramite $has\_no\_feedback$), , sia a livello globale, dove i dati del sensore IMU confermano la rotazione 
                                                             stimata in base agli RPM delle ruote (verificato tramite $isMotionConsistent$)\\ \hline
    



    \textbf{b2\_sup\_status}    & SUP\_DEGRADED             & Questo valore indica una discontinuità operativa della Board 2. Il supervisore della Board 1 monitora l'heartbeat (una variabile) del supervisore remoto, 
                                                              il quale incrementa il valore ogni volta che viene eseguito. Se la media degli ultimi 10 intervalli di aggiornamento superi i 40 ms, il sistema
                                                              segnala uno stato di degrado del supervisore di Board 2. \\\\
                                & SUP\_CRITICAL             & Se l'intervallo di tempo dall'ultimo aggiornamento dell'heartbeat del supervisore di Board2 supera i $120\,\mathrm{ms}$, imposta questo valore. \\\\
                                & SUP\_OK                   & Quando non ci sono ne condizioni critiche ne degradate imposta questo valore. \\\\ \hline
    



    \textbf{rx\_status}         & RX\_DEGRADED              & Questo valore identifica una comunicazione instabile. Sebbene il collegamento fisico sia attivo, fattori quali errori di checksum (CRC) 
                                                                o anomalie nella lunghezza dei pacchetti impediscono l'aggiornamento della variabile data\_last\_valid\_ms. Il sistema monitora la qualità
                                                                del link calcolando la media mobile degli ultimi 10 intervalli di ricezione valida; se tale media supera la soglia critica di $40$ ms, viene
                                                                segnalato il degrado della ricezione.\\\\
                                & RX\_CRITICAL              & Se l'intervallo di tempo dall'ultima ricezione corretta supera i $120\,\mathrm{ms}$, imposta questo valore. \\\\
                                & RX\_OK                    & Quando non ci sono ne condizioni critiche ne degradate imposta questo valore. \\ \hline



\end{longtable}





%%%%%%%%%%%%%%%%%%%%%%%%%%%%%%%%%%%%%%%%%%%%%%%%%%%%%%%%%%%%%%%%%%%%%%%%%%
\subsection{Decidere di far comandare la Board2}

Il sottosistema per decidere se autorizzare o meno la Board2 a muovere il rover è mostrato in Figura~\ref{fig:decidere_b2}.

\begin{figure}[H]
    \centering
    \includegraphics[width=0.8\textwidth]{images/SupervisoreB1/B1\_to\_B2\_command}
    \caption{\small{Chart per decidere se autorizzare o meno la Board2 a muovere il rover.}}
    \label{fig:decidere_b2}
\end{figure}

L'autorizzazione a Board 2 per il controllo del rover è regolata dalla variabile $give\_b2\_actuation$ secondo la seguente logica:

\begin{itemize}
    \item \textbf{give\_b2\_actuation = 1}:  Il relè è aperto e il movimento viene gestito da Board 2. Questa condizione si ha quando sono riscontrate anomalie $critiche$ in
                                                ricezione (textt\it{rx\_status = RX\_CRITICAL}), supervisore (textit{b2\_sup\_status = SUP\_CRITICAL}).
    \item \textbf{give\_b2\_actuation = 0}:  Il relè è chiuso e il movimento viene gestito da Board 1. Questa condizione si ha quando non sono riscontrate anomalie $critiche$ in
                                                ricezione (textit{rx\_status = RX\_OK}), supervisore (textit{b2\_sup\_status = SUP\_OK}).
\end{itemize}

Tale meccanismo agisce come un dispositivo di sicurezza hardware basato sullo stato del supervisore, come mostrato nelle Figure~\ref{fig:rele_b2} e ~\ref{fig:rele_b1}.

\begin{figure}[H]
    \begin{subfigure}{0.45\textwidth}
        \includegraphics[width=\textwidth]{images/SupervisoreB1/rele_board1}
        \caption{\small{Il comando di Board1 chiude il rele impedendo a Board2 di muovere il rover.}}
        \label{fig:rele_b1}
    \end{subfigure}
    \hfill
    \begin{subfigure}{0.45\textwidth}
        \includegraphics[width=\textwidth]{images/SupervisoreB1/rele_board2}
        \caption{\small{Board1 permette alla Board2 di muovere il rover.}}
        \label{fig:rele_b2}
    \end{subfigure}
\end{figure}

%\newpage

%%%%%%%%%%%%%%%%%%%%%%%%%%%%%%%%%%%%%%%%%%%%%%%%%%%%%%%%%%%%%%%%%%%%%%%%
\subsection{Costruzione maschere degradate e critiche}

\begin{figure}[H]
    \centering
    \includegraphics[width=0.8\textwidth]{images/SupervisoreB1/costruzione\_maschere}
    \caption{\small Chart per la costruzione delle maschere di errore critiche e degradate.}
    \label{fig:costruzione_maschere}
\end{figure}

Dagli stati critici e degradati rilevati nella parte di rilevamento dei faults, si costruiscono due maschere di errore a 8 bit, una per le anomalie critiche e una per quelle degradate.

Il bit di ogni maschera rappresenta un'anomalia specifica, come descritto nella Tabella~\ref{tab:fault_mapping}.

\begin{table}[H]
    \centering
    
    \begin{tabular}{|c|l|l|l|}
        \hline
        \textbf{Bit} & \textbf{Componente} & \textbf{critical\_mask} & \textbf{degraded\_mask} \\
        \hline
        0 & Temperatura & \textbf{1} se \textbf{TEMP\_HEALT\_CRITIC} & \textbf{1} se \textbf{TEMP\_HEALT\_DEG}\\
        \hline
        1 & Batteria    & \textbf{1} se \textbf{TEMP\_HEALT\_CRITIC} & \textbf{1} se \textbf{TEMP\_HEALT\_DEG}\\
        \hline
        2 & Ricevitore (RX) & \textbf{1} se \textbf{RX\_CRITICAL} & \textbf{1} se \textbf{RX\_DEGRADED}\\
        \hline
        3 & Ruota FL & \textbf{1} se \textbf{TEMP\_HEALT\_CRITIC} & \textbf{1} se \textbf{TEMP\_HEALT\_DEG}\\
        \hline
        4 & Ruota FR & \textbf{1} se \textbf{TEMP\_HEALT\_CRITIC} & \textbf{1} se \textbf{TEMP\_HEALT\_DEG}\\
        \hline
        5 & Ruota RL & \textbf{1} se \textbf{TEMP\_HEALT\_CRITIC} & \textbf{1} se \textbf{TEMP\_HEALT\_DEG} \\
        \hline
        6 & Ruota RR & \textbf{1} se \textbf{TEMP\_HEALT\_CRITIC} & \textbf{1} se \textbf{TEMP\_HEALT\_DEG} \\
        \hline
        7 & B2 Supervisor & \textbf{1} se \textbf{TEMP\_HEALT\_CRITIC} & \textbf{1} se \textbf{TEMP\_HEALT\_DEG} \\
        \hline
    \end{tabular}
    \caption{\small{Mappatura dei Bit nelle Fault Masks}}
\end{table}


\subsection{Calcolo riferimenti}
Il sottomodello per il calcolo dei riferimenti di velocità lineare e angolare del rover è mostrato in Figura~\ref{fig:calcolo_riferimenti}.
\begin{figure}[H]
    \centering
    \includegraphics[width=1\textwidth]{images/SupervisoreB1/decisione\_vel\_rot}
    \caption{\small Chart per il calcolo dei riferimenti di velocità lineare e angolare del rover.}
    \label{fig:calcolo_riferimenti}
\end{figure}

In particolare i riferimenti di velocità lineare e angolare del rover vengono calcolati in funzione dei comandi ricevuti dalla Board2 e delle condizioni di fault rilevate.
In assenza di anomalie, i riferimenti vengono calcolati direttamente a partire dai comandi ricevuti dalla Board2, in particolare, il riferimento di velocità lineare è calcolato 
a partire dal comando $y\_norm$ e il riferimento di velocità angolare è calcolato a partire dal comando $x\_norm$. Nel caso in cui siano rilevate anomalie, i riferimenti vengono
calcolati in modo da ridurre la velocità del rover o addirittura fermarlo, a seconda della gravità dell'anomalia rilevata. I riferimenti venogno scalati come mostrato 
in Figura~\ref{fig:scaling_riferimenti}.
\begin{figure}[H]
    \centering
    \includegraphics[width=0.8\textwidth]{images/SupervisoreB1/scaling\_Riferimenti}
    \caption{\small Scaling dei riferimenti in base alla gravità dell'anomalia rilevata.}
    \label{fig:scaling_riferimenti}
\end{figure}
Nella figura~\ref{fig:scaling_riferimenti} è mostrato come i riferimenti di velocità lineare e angolare vengono scalati in base alla gravità dell'anomalia rilevata, in 
particolare utilizzando la variabile $safety\_status$ che viene calcolata a partire dalle maschere di errore critiche e degradate di entrambe le Board, come mostrato in figura ~\ref{fig:calcolo_safety_status}.
\begin{figure}[H]
    \centering
    \includegraphics[width=0.8\textwidth]{images/SupervisoreB1/calcolo\_safety\_status}
    \caption{\small Calcolo della variabile safety\_status a partire dalle maschere di errore critiche e degradate di entrambe le Board.}
    \label{fig:calcolo_safety_status}
\end{figure}










































%%%%%%%%%%%%%%%%%%%%%%%%%%%%%%%%%%%%%%
\newpage
\section{Supervisore Board 2}

\subsection{Panoramica generale}
Il supervisore della Board 2 è implementato come un modulo Simulink denominato \textbf{SupervisorB2}, il cui schema a blocchi è illustrato in Figura~\ref{fig:supervisorB2_block}.
\begin{figure}[H]
    \centering
    \includegraphics[width=0.8\textwidth]{images/SupervisoreB2/SupervisoreBoard2_InputOutput}
    \caption{Schema a blocchi del modulo SupervisorB2.}
    \label{fig:supervisorB2_block} 
\end{figure}
Il suo compito è quello di decidere il comando di direzione \textit{B2Decision} del rover, verificare se ci sono incoerenze tra l'angolo di direzione stimato dalla velocità 
delle ruote e l'IMU e, nel caso sia autorizzato, calcolare i riferimenti di velocità lineare e angolare del rover.

In particolare, esso è composto da 4 parti principali:
\begin{itemize}
    \item \textbf{Rilevazione Faults}: si occupa di rilevare eventuali anomalie nell'ESP32, nell'IMU, nella ricezione dei dati da Board1.
    \item \textbf{Aggregazione Fualt}: aggrega le anomalie rilevate nella parte di $rilevazione faults$ e le codifica in due maschere di errore (critica e degradata).
    \item \textbf{Decidere se attuare}: decide se inviare il segnale di attuazione UART al rele.
    \item \textbf{Rilvamento ostacoli}: si occupa di rilevare la presenza di ostacoli tramite i sensori sonar e, e influenzare il comando di movimento del rover in caso di ostacoli rilevati.
    \item \textbf{Calcolo Riferimenti}: se autorizzato, calcola i riferimenti di velocità lineare e angolare del rover in base ai comandi ricevuti dall'utente.
\end{itemize}
Nel seguito verranno descritti i segnali di input e output del supervisore, successivamente verranno descritte le quattro parti principali del supervisore descritte sopra.





\subsection{Input}
I segnali in input che riceve sono:

\begin{itemize}  
    \item \textbf{Board1\_Data}: è l'ultimo snapshot di ricezione ($RxSnapshot$) aggiornato dal task di ricezione di Board2. I segnali utilizzati dal supervisore di Board 2 sono:
    \begin{itemize}
        \item \textit{wheel\_speed\_rpm(i)}: è la velocità in RPM di ogni ruota.
            
        \item \textit{critical\_mask}: è una maschera di errore a 8 bit calcolata da Board 1, in cui ogni bit indica la presenza di un'anomalia \textit{critica} specifica.
  
        \item \textit{data\_last\_valid\_ms}: rappresenta il tempo in millisecondi dell'ultimo dato valido ricevuto da Board 1, utilizzato per rilevare eventuali timeout nella ricezione.
        
        
    \end{itemize}
    \item \textbf{IMU}: è l'ultimo snapshot ($IMUSnapshot$) di rilevamento dell'angolo yaw. la struttura dati è la seguente:
        \begin{figure}[H]
            \centering
            \includegraphics[width=0.7\textwidth]{images/SupervisoreB2/IMU_snapshot}
            \caption{Struttura dati IMU\_snapshot.}
            \label{fig:board_health_struct}
        \end{figure}
    \item \textbf{Sonars}: è l'ultimo snapshot delle velocità delle ruote ($wheel\_speed\_rpm$) e di eventuali anomalie riscontrate in uno dei motori ($has\_no\_feedback$). La struttura dati è la seguente:
        \begin{figure}[H]
            \centering
            \includegraphics[width=0.7\textwidth]{images/SupervisoreB2/Sonars_snapshot}
            \caption{Struttura dati sonar\_snapshot.}
            \label{fig:encoder_struct}
        \end{figure}
    \item \textbf{now\_ms}: Rappresenta il tempo corrente in millisecondi.
    \item \textbf{last\_valid\_b1\_ms}: Rappresenta il tempo in millisecondi dell'ultimo volta che si è rilevato che il supervisore di Board 1 sia funzionante.
\end{itemize}













\subsection{Output}
I segnali in output che fornisce sono:
\begin{itemize}
    \item \textbf{B2Decision}: rappresenta il riferimento di velocità lineare del rover, in m/s.    
    \item \textbf{critical\_mask \& degraded\_mask}: sono due mashere di errore a 8 bit, in cui ogni bit indica la presenza 
    di un'anomalia \textit{critica} (da cui critical\_mask) o \textit{degradata} (da cui degraded\_mask) specifica, come descritto 
    nella Tabella~\ref{tab:fault_mapping_B2}. Nella sezione di $Rilevazione Faults$ viene descritto nello specifico cosa indicano i singoli bit di queste maschere di errore.
    \begin{table}[H]
        \centering
        \begin{tabular}{lll}
        \hline
        \textbf{Bit} & \textbf{ID Segnale} & \textbf{Descrizione dell'Anomalia} \\ \hline
        0 & ESP\_CRI/DEG  & Timeout della comunicazione con ESP32 \\
        1 & IMU\_CRI/DEG  & Timeout della comunicazione con il sensore IMU \\
        2 & RX\_CRI/DEG   & Comunicazione con Board 1 lenta o assente\\
        3 & B1\_SUP\_CRI/DEG & Supervisore Board 1 lento o assente \\ 
        4-7 & \textit{Reserved} & Bit riservati per espansioni \\ \hline
        \end{tabular}
        \caption{Mappatura della maschera di errore della Board 2}
        \label{tab:fault_mapping_B2}
    \end{table}





    \item \textbf{isMotionConsistent}: rappresnta un segnale booleano che indica se c'è coerenza tra la rotazione misurata dalla IMU e quella stimata dalla velocità dei motori.
    \item \textbf{autorized\_to\_send\_command}: rappresenta un segnale booleano che indica se la Board2 può inviare comandi di movimento al rover.
    \item \textbf{v\_ref\_actuation}: rappresenta il riferimento di velocità lineare del rover, in RPM, calcolato dal supervisore di Board 2 in base ai comandi ricevuti dall'utente.
                                      Tale riferimento viene attuato dalla Board 2 solo se il segnale $authorized\_to\_send\_command$ è attivo.
    \item \textbf{omega\_ref\_actuation}: rappresenta il riferimento di velocità angolare del rover, in rad/s, calcolato dal supervisore di Board 2 in base ai comandi ricevuti dall'utente.
                                          Tale riferimento viene attuato dalla Board 2 solo se il segnale $authorized\_to\_send\_command$ è attivo.
    \item \textbf{actuate\_emergency\_stop}: rappresenta un segnale booleano che indica se il rover deve essere fermato immediatamente.
                                             Tale segnale viene attivato solo se il segnale $authorized\_to\_send\_command$ è attivo.
    \item \textbf{current\_action}: rappresenta un segnale che indica l'azione attualmente intrapresa dal supervisore
\end{itemize}


















\subsection{Rilevazione Faults}
\begin{figure}[H]
    \centering
    \includegraphics[width=0.8\textwidth]{images/SupervisoreB2/Rilevamento_Faults}
    \caption{Chart di gestione dei faults.}
    \label{fig:gestione_faults_b2}
\end{figure}

L'output di questo sottosistema è rappresentato da 5 segnali, ognuno dei quali indica la presenza o meno di un'anomalia \textit{critica} o
\textit{degradata} specifica proveniente dalla lettura dei segnali dei comandi utente ricevuti dall' ESP32, del sensore IMU e della comunicazione con Board1. Di seguito si
descrivono i valori che questi segnali possono assumere e le condizioni che portano a tali valori.

\begin{longtable}{ll p{5cm}} 
    \caption{Stati degradato e critico per Board 2} \label{tab:fault_table_b2} \\
    \hline
    \textbf{OUTPUT} & \textbf{VALORE} & \textbf{SIGNIFICATO} \\
    \hline
    \endfirsthead % Cosa appare nella prima pagina

    \hline
    \textbf{OUTPUT} & \textbf{VALORE} & \textbf{SIGNIFICATO} \\
    \hline
    \endhead % Cosa appare in cima alle pagine successive (opzionale)

    
    \endfoot % Cosa appare in fondo a ogni pagina (opzionale)

    
    % --- Inizio dati tabella ---
    \textbf{esp\_status}        & ESP\_DEGRADED              & Se la media mobile degli intervalli di ricezione dati dall'ESP32 supera la soglia di $40\,\mathrm{ms}$. \\
                                & ESP\_CRITICAL              & Se la media mobile degli intervalli di ricezione dati dall'ESP32 supera la soglia di $120\,\mathrm{ms}$. \\
                                & ESP\_OK                    & Quando la frequenza media di aggiornamento rientra nei limiti nominali di sicurezza stabiliti. \\ \hline
    
    \textbf{imu\_status}        & IMU\_DEGRADED             & Se la media mobile degli intervalli di aggiornamento dell'angolo di yaw dell'IMU supera i $40\,\mathrm{ms}$. \\
                                & IMU\_CRITICAL             & Se la media mobile degli intervalli di aggiornamento dell'angolo di yaw dell'IMU supera i $120\,\mathrm{ms}$. \\
                                & IMU\_OK                   & Quando la frequenza media dei dati inerziali rientra nei parametri nominali di funzionamento. \\ \hline 

    \textbf{b1\_sup\_status}    & SUP\_DEGRADED             & Questo valore indica una discontinuità operativa della Board 1. Il supervisore della Board 2 monitora l'heartbeat (una variabile) del supervisore remoto, 
                                                              il quale incrementa il valore ogni volta che viene eseguito. Se la media degli ultimi 10 intervalli di aggiornamento superi i 40 ms, il sistema
                                                              segnala uno stato di degrado del supervisore di Board 1. \\\\
                                & SUP\_CRITICAL             & Se l'intervallo di tempo dall'ultimo aggiornamento dell'heartbeat del supervisore di Board1 supera i $120\,\mathrm{ms}$, imposta questo valore. \\\\
                                & SUP\_OK                   & Quando non ci sono ne condizioni critiche ne degradate imposta questo valore. \\\\ \hline
                                
    \textbf{rx\_status}         & RX\_DEGRADED              & Questo valore identifica una comunicazione instabile. Sebbene il collegamento fisico sia attivo, fattori quali errori di checksum (CRC) 
                                                                o anomalie nella lunghezza dei pacchetti impediscono l'aggiornamento della variabile data\_last\_valid\_ms. Il sistema monitora la qualità
                                                                del link calcolando la media mobile degli ultimi 10 intervalli di ricezione valida; se tale media supera la soglia critica di $40$ ms, viene
                                                                segnalato il degrado della ricezione.\\\\
                                & RX\_CRITICAL              & Se l'intervallo di tempo dall'ultima ricezione corretta supera i $120\,\mathrm{ms}$, imposta questo valore. \\\\
                                & RX\_OK                    & Quando non ci sono ne condizioni critiche ne degradate imposta questo valore. \\ \hline
    
    \textbf{isMotionConsistent} & 0                         & Indica che c'è una discrepanza significativa tra la rotazione misurata dalla IMU e quella stimata dalla velocità dei motori, suggerendo un possibile guasto o malfunzionamento. \\
                                & 1                         & Indica che i dati della IMU e le stime basate sulla velocità dei motori sono coerenti, suggerendo un funzionamento normale del sistema di movimento. \\ \hline
\end{longtable}



\subsection{Costruzione maschere degradate e critiche}

\begin{figure}[H]
    \centering
    \includegraphics[width=0.8\textwidth]{images/SupervisoreB2/costruzione_maschere}
    \caption{Chart per l'aggregazione delle faults.}
    \label{fig:aggregazione_faults_b2}
\end{figure}


Dagli stati critici e degradati rilevati nella parte di gestione faults, si costruiscono due maschere di errore a 8 bit, una per le anomalie critiche e una per quelle degradate.

Il bit di ogni maschera rappresenta un'anomalia specifica, come descritto nella Tabella~\ref{tab:fault_mapping}.

\begin{table}[H]
    \centering
    
    \begin{tabular}{|c|l|l|l|}
        \hline
        \textbf{Bit} & \textbf{Componente} & \textbf{critical\_mask} & \textbf{degraded\_mask} \\
        \hline
        0 & ESP32 & \textbf{1} se \textbf{ESP\_CRITICAL} & \textbf{1} se \textbf{ESP\_DEGRADED}\\
        \hline
        1 & IMU & \textbf{1} se \textbf{IMU\_CRITICAL} & \textbf{1} se \textbf{IMU\_DEGRADED}\\
        \hline
        2 & Ricezione da Board 1 & \textbf{1} se \textbf{RX\_CRITICAL} & \textbf{1} se \textbf{RX\_DEGRADED}\\
        \hline
        3 & Supervisore Board 1 & \textbf{1} se \textbf{SUP\_CRITICAL} & \textbf{1} se \textbf{SUP\_DEGRADED}\\
        \hline
        4-7 & Reserved & Bit riservati per espansioni future & Bit riservati per espansioni future\\
        \hline
    \end{tabular}
    \caption{\small Mappatura dei Bit nelle Fault Masks di Board 2}
    \label{tab:fault_mapping}
\end{table}


\subsection{Decidere se attuare}


\begin{figure}[H]
    \centering
    \includegraphics[width=0.8\textwidth]{images/SupervisoreB2/decidere_se_attuare}
    \caption{Chart per decidere se autorizzare o meno la Board2 a muovere il rover.}
    \label{fig:decidere_attuare_b2}
\end{figure}

L'autorizzazione a Board 2 per il controllo del rover è regolata dalla variabile $authorized\_to\_send\_command$ secondo la seguente logica:
\begin{itemize}
    \item \textbf{authorized\_to\_send\_command = 1}:  Board 2 è autorizzata a inviare comandi di movimento al rover. Questa condizione si ha quando sono riscontrate anomalie $critiche$ in
                                                ricezione (\textit{rx\_status = RX\_CRITICAL}), supervisore (\textit{b2\_sup\_status = SUP\_CRITICAL}), oppure quando è la stessa Board 1 che 
                                                chiede a Board 2 di prendere il controllo del rover.
    \item \textbf{authorized\_to\_send\_command = 0}:  Board 2 non è autorizzata a inviare comandi di movimento al rover. Questa condizione si ha quando non sono riscontrate anomalie $critiche$ in
                                                ricezione (\textit{rx\_status = RX\_OK}), supervisore (\textit{b2\_sup\_status = SUP\_OK}) e quando non è la stessa Board 1 a chiedere a Board 2 
                                                di prendere il controllo del rover.

\end{itemize}







\subsection{Rilevamento ostacoli}
Come da specifiche, il comportamento del rover in presenza di ostacoli, deve essere regolato sulla base delle condizioni in cui può trovarsi:
\begin{enumerate}
    \item \textit{Stato non degradato}
    \begin{itemize}
        \item \textbf{Distanza dell'ostacolo $\le$ 70 cm:} il rover deve fermarsi immediatamente per evitare collisioni.
        \item \textbf{Ostacolo a distanza $>$ 100 cm in movimento tra due sonar:} il rover deve determinare la direzione dell'ostacolo e deve deviare il 
        percorso di conseguenza in direzione del sonar che per prima ha rilveato l'ostacolo.
    \end{itemize}
    \item \textit{Stato degradato}
    \begin{itemize}
        \item \textbf{Distanza dell'ostacolo $\le$ 300 cm:} il rover deve fermarsi immediatamente per evitare collisioni.
    \end{itemize}
\end{enumerate}

In seguito verranno mostrati i chart realizzati per la gestione delle due casistiche.


\subsubsection{Gestione ostacoli con sistema in stato \textit{non degradato}}

\begin{figure}[H]
    \centering
    \includegraphics[width=0.9\textwidth]{images/SupervisoreB2/GestioneRilevamentoOstacoli/GestioneOstacoliB2}
    \caption{Logica di gestione degli ostacoli in stato non degradato.}
    \label{fig:logica_gestione_ostacoli}
\end{figure}

Il chart per la gestione degli ostacoli in stato non degradato è mostrato in Figura~\ref{fig:chart_non_degradato}.

\begin{figure}[H]
    \centering
    \includegraphics[width=0.9\textwidth]{images/SupervisoreB2/GestioneRilevamentoOstacoli/GestioneOstacoli_StatoNonDegradato}
    \caption{Chart di gestione ostacoli in stato non degradato.}
    \label{fig:chart_non_degradato}
\end{figure}

In particolare, il chart è composto da 2 stati paralleli: \textit{Sonars} e \textit{B2Decisione}

\begin{figure}[H]
    \centering
    \begin{subfigure}{0.45\textwidth}
        \includegraphics[width=\textwidth]{images/SupervisoreB2/GestioneRilevamentoOstacoli/SonarsState}
        \caption{Stato parallelo Sonars}
    \end{subfigure}
    \hfill
    \begin{subfigure}{0.45\textwidth}
        
        \includegraphics[width=\textwidth]{images/SupervisoreB2/GestioneRilevamentoOstacoli/B2DecisionState}
        \caption{Stato parallelo B2Decisione}
    \end{subfigure}
    
    \caption{Stati paralleli del chart in stato non degradato.}
    \label{fig:stati_paralleli}
\end{figure}



\begin{enumerate}
    \item \textbf{\textit{B2Decisions}}: 
        Lo stato parallelo \textit{B2Decisione} è dipendente dallo stato $Sonars$ in quanto le sue transizioni vengono attivate da segnali provenienti da $Sonars$. In base ai segnali
        ricevuti, è capace di settare la variabile di output del chart, variabile che indica la decisione presa dal supervisore.
        Quindi, in questo stato si determina l'output del chart, che è un numero che varia da $0$ a $4$.
        Le azioni possibili includono l'arresto immediato del rover o la deviazione del percorso in base alla posizione dell'ostacolo. In quest'ultimo caso, la deviazione dura fintanto che il rover non ruota
        di 45° rispetto alla direzione iniziale, verso la direzione del sonar che per primo ha rilevato l'ostacolo.
    \item \textbf{\textit{Sonars}}: 
        All'interno di questo stato parallelo sono presenti altri \textit{3} stati paralleli.
        \begin{enumerate}
            \item \textbf{\textit{distance\_70}}: Rappresenta la condizione in cui uno dei sonar rileva un ostacolo a una distanza $\leq 70$ cm.
                \begin{figure}[H]
                \centering
                \includegraphics[width=0.9\textwidth]{images/SupervisoreB2/GestioneRilevamentoOstacoli/distanceLt_70}
                \caption{Stato parallelo per la gestione di un ostacolo a distanza $\le$ 70 cm.}
                \label{fig:sonars_non_degradato}
                \end{figure}
        
        In questo stato quando uno dei sonar rileva un ostacolo a una distanza inferiore o uguale a 70 cm, viene attivata una transizione che porta allo stato di arresto immediato del rover.
        In particolare, quando un sonar rileva la presenza di un ostacolo a distanza $\leq 70$ cm, viene inviato un segnale \textbf{Emergency} allo stato parallelo \textit{B2Decisione} per fermare il rover. 
        B2Decision utilizza questo segnale per portarsi nello stato in cui l'output del chart prevede lo stop.
            
            \item \textbf{\textit{distance\_gt70 ---- timers}}: 
                Questi due stati insieme permettono il rilevamento di un ostacolo in movimento tra le coppie di sonar 
                \begin{itemize}
                    \item \textit{S1-S2} (tra sonar di sinistra e sonar centrale)
                    \item \textit{S2-S1} (tra sonar centrale e sonar di sinistra)
                    \item \textit{S2-S3} (tra sonar centrale e sonar di destra)
                    \item \textit{S3-S2} (tra sonar di destra e sonar centrale)  
                \end{itemize}
                
                \begin{figure}[H]
                    \centering
                    \includegraphics[width=0.9\textwidth]{images/SupervisoreB2/GestioneRilevamentoOstacoli/distancegt_70_and_timers}
                    \caption{Stato parallelo per la gestione di un ostacolo in movimento a distanza $>$ 70 cm.}
                    \label{fig:stati_gestione_timers}
                \end{figure}
                Nello stato \textit{distance\_gt70} sono presenti $3$ stati paralleli, \textit{Waiting\_S1}, \textit{Waiting\_S2}, \textit{Waiting\_S3}, uno per ogni sonar.
                
                Di seguito si analizza la dinamica di rilevamento di un ostacolo che si sposta dal sonar $S1$ verso il sonar $S2$.
                Tale logica è da considerarsi valida per ogni coppia di sensori precedentemente elencata. 
                
                Si assume, come condizione necessaria, l'assenza di ostacoli a una distanza inferiore a $70$ cm; in caso contrario, il sistema non procederebbe al rilevamento di oggetti in movimento.
                
                \begin{enumerate}
                    \item \textbf{Attivazione ($S1$)}: Quando il sonar $S1$ rileva un oggetto entro il range $100$–$300$ cm, la variabile $obj1Detected$ viene impostata a $1$ \textbf{(fronte di salita)}.
                    \item \textbf{Transizione e Timing}: Nel momento in cui l'oggetto esce dal campo d'azione di $S1$, la variabile $obj1Detected$ torna a $0$ \textbf{(fronte di discesa)}.
                    Contestualmente, lo stato timerSonar12 del modulo timers avvia un conteggio di $3$ secondi.
                    \item \textbf{Verifica ($S2$)}: Se il sonar $S2$ rileva l'ostacolo (sempre tra $100$ e $300$ cm) entro la finestra temporale dei $3$ secondi, viene inviato il segnale $moveToS1$ allo 
                    stato parallelo $B2Decision$. Qualora il timer scada senza alcun rilevamento da parte di $S2$, non viene trasmesso alcun segnale.
                \end{enumerate}
            \end{enumerate}
\end{enumerate}

\subsubsection{Gestione ostacoli con sistema in stato $degradato$}

Il chart per la gestione degli ostacoli in stato degradato è mostrato in Figura~\ref{fig:chart_degradato}.

\begin{figure}[H]
    \centering
    \includegraphics[width=0.9\textwidth]{images/SupervisoreB2/GestioneRilevamentoOstacoli/GestioneOstacoli_StatoDegradato}
    \caption{Chart di gestione ostacoli in stato degradato.}
    \label{fig:chart_degradato}
\end{figure}



In questo caso, la logica di gestione degli ostacoli è semplificata rispetto allo stato non degradato.
Infatti, l'unica condizione considerata è la presenza di un ostacolo a una distanza inferiore o uguale a $300$ cm.
Quando uno dei sonar rileva un ostacolo entro questo range, viene attivata una transizione che porta l'uscita del supervisore all'arresto immediato del rover.


\subsection{Attuazione}
L'attuazione della Board 2 avviene esclusivamente in modalità degradata, coerentemente con il suo ruolo di Slave. Nello specifico, la Board 2 è autorizzata a prendere il
controllo del rover solo quando si verificano determinate condizioni critiche sulla Board 1: il riscontro di anomalie gravi nella ricezione dei dati (rx\_status = RX\_CRITICAL),
errori critici nel modulo di supervisione (b2\_sup\_status = SUP\_CRITICAL), oppure nel caso in cui la stessa Board 1 invii una richiesta esplicita di cessione del controllo.
Ciò indica che, nonostante la Board 2 sia ad attuare, la comunicazione tra le due board potrebbe essere ancora attiva, e in particolare Board 1 invia il proprio payload che include 
velocità delle ruote e maschere di errore.
\begin{figure}[H]
    \centering
    \includegraphics[width=0.8\textwidth]{images/SupervisoreB2/attuazione_Board2}
    \caption{Chart per la gestione dell'attuazione del comando di movimento del rover}
    \label{fig:attuazione_b2}
\end{figure}

Se la comunicazione tra le due board è attiva, il supervisore verifica che la critical mask della Board 1 non indichi anomalie critiche che non siano quelle di ricezione o supervisione,





\end{document}